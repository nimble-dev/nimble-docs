\documentclass[10pt,landscape]{article}
%% Cheat sheet formatting thanks to http://tex.stackexchange.com/questions/8827/preparing-cheat-sheets from user Dror
\usepackage{multicol}
\usepackage{calc}
\usepackage{ifthen}
\usepackage[landscape]{geometry}
\usepackage{amsmath,amsthm,amsfonts,amssymb}
\usepackage{color,graphicx,overpic}
\usepackage{hyperref}

\def\cd#1{\texttt{#1}}
\newcommand{\ver}{0.6-4}

\pdfinfo{
  /Title (nimbleCheatSheet.pdf)
  /Creator (TeX)
  /Producer ()
  /Author (NIMBLE development team)
  /Subject (NIMBLE)
  /Keywords ()}

% This sets page margins to .5 inch if using letter paper, and to 1cm
% if using A4 paper. (This probably isn't strictly necessary.)
% If using another size paper, use default 1cm margins.
\ifthenelse{\lengthtest { \paperwidth = 11in}}
    { \geometry{top=.5in,left=.5in,right=.5in,bottom=.5in} }
    {\ifthenelse{ \lengthtest{ \paperwidth = 297mm}}
        {\geometry{top=1cm,left=1cm,right=1cm,bottom=1cm} }
        {\geometry{top=1cm,left=1cm,right=1cm,bottom=1cm} }
    }

% Turn off header and footer
\pagestyle{empty}

% Redefine section commands to use less space
\makeatletter
\renewcommand{\section}{\@startsection{section}{1}{0mm}%
                                {-1ex plus -.5ex minus -.2ex}%
                                {0.5ex plus .2ex}%x
                                {\normalfont\large\bfseries}}
\renewcommand{\subsection}{\@startsection{subsection}{2}{0mm}%
                                {-1explus -.5ex minus -.2ex}%
                                {0.5ex plus .2ex}%
                                {\normalfont\normalsize\bfseries}}
\renewcommand{\subsubsection}{\@startsection{subsubsection}{3}{0mm}%
                                {-1ex plus -.5ex minus -.2ex}%
                                {1ex plus .2ex}%
                                {\normalfont\small\bfseries}}
\makeatother

% Define BibTeX command
\def\BibTeX{{\rm B\kern-.05em{\sc i\kern-.025em b}\kern-.08em
    T\kern-.1667em\lower.7ex\hbox{E}\kern-.125emX}}

% Don't print section numbers
\setcounter{secnumdepth}{0}


\setlength{\parindent}{0pt}
\setlength{\parskip}{0pt plus 0.5ex}

%My Environments
\newtheorem{example}[section]{Example}
% -----------------------------------------------------------------------

\begin{document}
\raggedright
\footnotesize
\begin{multicols}{3}


% multicol parameters
% These lengths are set only within the two main columns
%\setlength{\columnseprule}{0.25pt}
\setlength{\premulticols}{1pt}
\setlength{\postmulticols}{1pt}
\setlength{\multicolsep}{1pt}
\setlength{\columnsep}{2pt}

\begin{center}
     \Large{\underline{NIMBLE \ver\ cheat sheet}} \\
\end{center}

\section{Writing BUGS models}
\cd{bare bones example}
\subsection{Common mistakes}
\begin{itemize}
\item Use \cd{X[1:n]} or \cd{X[]}, not just \cd{X}.
\item Default parameter ordering follows BUGS, not R.
\item etc.
\end{itemize}

\subsection{distributions}
\begin{multicols}{2}
\begin{itemize}
\item dnorm(mean, sd)
\item dnorm(mean, tau)
\item etc.
\end{itemize}
\end{multicols}

\subsection{functions}
\begin{multicols}{2}
\begin{itemize}
\item inprod(vector, vector)
\item etc.
\end{itemize}
\end{multicols}

\section{Running algorithms}

\subsection{MCMC}

\subsection{Particle Filters}

\section{Writing nimbleFunctions}
\subsection{Without setup}

\subsection{With setup}

\subsection{Type declarations}

\begin{itemize}
\item (logical, integer, double-precision) scalar: (\cd{logical(0)}, \cd{integer(0)}, \cd{double(0)})
\item (logical, integer, double-precision) vector: (\cd{logical(1)}, \cd{integer(1)}, \cd{double(1)})
\item etc.
\end{itemize}


\subsection{Available functions}

\subsection{Querying model structure}
\begin{itemize}
\item \cd{model\$getDependencies(nodes)}
\item etc.
\end{itemize}

\subsection{Using models}
\begin{itemize}
\item \cd{model\$calculate(nodes)}
\item etc.
\end{itemize}

\section{Section 3}
Etc.

% You can even have references
\rule{0.3\linewidth}{0.25pt}
\scriptsize
\bibliographystyle{abstract}
\bibliography{refFile}
\end{multicols}
\end{document}