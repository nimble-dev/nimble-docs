% \begin{longtable}[c]{ll}
%  \caption{Basic R manipulation functions in NIMBLE.} \label{table:functions-coreR} \\
%   \hline
  Function & Comments (differences from R) \\
\hline \hline \\
\endhead
\cd{c()} & No ``\cd{recursive}'' argument. \\
\cd{rep()} & No ``\cd{rep.int}'' or ``\cd{rep\_len}'' arguments. \\
\cd{seq()} and `\cd{:}' & Negative integer sequences from `\cd{:}', e.g. \cd{2:1}, do not work. \\
\cd{which()} & No ``\cd{arr.ind}'' or ``\cd{useNames}'' arguments. \\
\cd{diag()} & \parbox{5in}{Works like R in three ways:
\cd{diag(vector)} returns a matrix with \cd{vector} on the diagonal;
\cd{diag(matrix)} returns the diagonal vector of \cd{matrix};
\cd{diag(n)} returns an $n \times n$ identity matrix.  No ``\cd{nrow}'' or ``\cd{ncol}'' arguments.}\\
\cd{diag()<-} & Works for assigning the diagonal vector of a matrix.\\
\cd{dim()} & Works on a vector as well as higher-dimensional arguments. \\
\cd{length()} & \\
\cd{numeric()} & Allows additional arguments to control initialization. \\
\cd{logical()} & Allows additional arguments to control initialization. \\
\cd{integer()} & Allows additional arguments to control initialization. \\
\cd{matrix()} & Allows additional arguments to control initialization. \\
\cd{array()} & Allows additional arguments to control initialization. \\
indexing & Arbitrary integer and logical indexing is supported.\\
  \hline \\

%  \end{longtable}
% }